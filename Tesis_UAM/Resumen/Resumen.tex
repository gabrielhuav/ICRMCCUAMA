%!TEX root = ../ICR.tex
\thispagestyle{plain}  
\vspace{0.9cm}
\textbf{{\Large Resumen}}
\vspace{0.9cm}

Este proyecto aborda el desafío de la detección de fraude digital y noticias falsas en español mediante la aplicación y comparación de dos metodologías de inteligencia artificial. La primera explora el uso de algoritmos metaheurísticos, incluyendo Recocido Multiarranque (MSA), Búsqueda Dispersa (SS), Búsqueda en Vecindades Variables (VNS), Algoritmo Genético (GA) y Optimización por Enjambre de Partículas (PSO), sobre una representación de Bolsa de Palabras (BoW). La segunda, adoptada tras los hallazgos iniciales, se basa en el ajuste fino (fine-tuning) de un modelo de lenguaje Transformer pre-entrenado (DistilBERT). Para el entrenamiento, se construyó un corpus unificando cuatro conjuntos de datos públicos en español y datos extraídos mediante extracción web del portal satírico "El Deforma", resultando en más de 61,000 noticias. Se implementó un cuidadoso proceso de calibración de hiperparámetros para ambos enfoques, utilizando una división de datos estratificada de 70\% para entrenamiento, 10\% para validación y 20\% para pruebas. El rendimiento fue evaluado con métricas como Exactitud, Precisión, Exhaustividad y F1-Score. Finalmente, el modelo Transformer, que demostró una eficacia superior, fue integrado en una aplicación web funcional desarrollada con Flask y contenerizada con Docker, capaz de analizar URLs en tiempo real. Los resultados validan la metodología de ajuste fino como una solución de vanguardia para combatir la desinformación, superando a los enfoques metaheurísticos en esta tarea.

\vspace{0.9cm}
\textbf{Palabras clave:} Detección de noticias falsas, fraude digital, modelos de lenguaje, Transformers, DistilBERT, algoritmos metaheurísticos, procesamiento de lenguaje natural.

\shipout\null
\newpage

\thispagestyle{plain}  
\vspace{0.9cm}
\textbf{{\Large Abstract}}
\vspace{0.9cm}

This project addresses the challenge of digital fraud and fake news detection in Spanish through the application and comparison of two artificial intelligence methodologies. The first explores the use of metaheuristic algorithms, including Multi-Start Simulated Annealing (MSA), Scatter Search (SS), Variable Neighborhood Search (VNS), Genetic Algorithm (GA), and Particle Swarm Optimization (PSO), applied to a Bag-of-Words (BoW) representation. The second, adopted following initial findings, is based on fine-tuning a pre-trained Transformer language model (DistilBERT). For training, a corpus was constructed by unifying four public Spanish datasets and data extracted through web scraping from the satirical portal "El Deforma", resulting in over 61,000 news articles. A careful hyperparameter calibration process was implemented for both approaches, using a stratified data split of 70\% for training, 10\% for validation, and 20\% for testing. Performance was evaluated using metrics such as Accuracy, Precision, Recall, and F1-Score. Finally, the Transformer model, which demonstrated superior efficacy, was integrated into a functional web application developed with Flask and containerized with Docker, capable of analyzing URLs in real-time. The results validate the fine-tuning methodology as a state-of-the-art solution for combating misinformation, outperforming metaheuristic approaches in this task.

\vspace{0.9cm}
\textbf{Keywords:} Fake news detection, digital fraud, language models, Transformers, DistilBERT, metaheuristic algorithms, natural language processing.

\shipout\null
